\documentclass[12pt]{article}
\usepackage[a4paper, total={6in, 9in}]{geometry}
\usepackage{graphicx}
\graphicspath{ {./images/output/} }
\usepackage{caption}
\usepackage[english]{babel}
\usepackage{titling}
\usepackage{float}
% \usepackage{amsmath}
% \usepackage{minted}
% \usepackage{multicol}
% \usepackage{array}
% \usepackage{setspace}
% \usepackage{placeins}

% \usepackage{lipsum}

\title{Basics of Oscilloscope and Signal Generator}
\author{}
\date{}

\pagenumbering{gobble}
\begin{document}
\vspace*{\fill}
\begin{center}

    \emph{Heaven's Light is Our Guide} \\
    \textbf{Rajshahi University of Engineering and Technology} \\

    \begin{figure}[H]
        \centering
        \includegraphics[scale=.34]{images/RUET_logo.png}
        \label{fig:ruet_logo}
    \end{figure}
    \vspace{5mm}

    \textbf{Course Code}\\
    ECE 3206\\
    \vspace{3mm}
    \textbf{Course Title}\\
    Industrial Electronics Sessional

    \vspace{5mm}
    \textbf{Experiment Date:} {November 24, 2024},\\
    \textbf{Submission Date:} {December 9, 2024}\\

    \vspace{5mm}
    \textbf{Lab Report 1: \\
        Basics of Oscilloscope and Signal Generator}

    \vspace{15mm}

    \begin{tabular}{c|c}
        \textbf{Submitted to} & \textbf{Submitted by} \\
        Md. Faysal Ahamed     & Md. Tajim An Noor     \\
        Lecturer              & Roll: 2010025         \\
        Dept of ECE, Ruet     &                       \\
    \end{tabular}

\end{center}
\vspace*{\fill}


\pagebreak

\tableofcontents

\pagebreak
\pagenumbering{arabic}
\maketitle

\section*{Theory}
\addcontentsline{toc}{section}{Theory}

\subsection*{Oscilloscope}
\addcontentsline{toc}{subsection}{Oscilloscope}
An oscilloscope is an electronic test instrument that allows observation of constantly varying signal voltages, typically as a two-dimensional plot of one or more signals as a function of time. It is widely used in electronics, telecommunications, and engineering to measure and visualize waveforms.\\\\
\textbf{Features of Oscilloscope:}
\begin{itemize}
    \item[] \textbf{Waveform Visualization:} Shows the shape of electrical signals such as voltage \& currents.
    \item[] \textbf{Measurement:} Provides tools to measure signal properties, such as amplitude, frequency, rise time, peak voltage, periodicity, and many other properties.
    \item[] \textbf{Trigger:} Allows stable visualization of repetitive signals by syncing the display to a specific point in the waveform.\\
\end{itemize}
\textbf{Types:}
\begin{itemize}
    \item[] \textbf{Analog Oscilloscope:} Displays signals in real time using cathode ray tubes.
    \item[] \textbf{Digital Oscilloscope:} Uses digital storage to record and display signals.
    \item[] \textbf{Mixed Signal Oscilloscope (MSO):} Combines analog and digital signal analysis.
\end{itemize}

\begin{figure}[H]
    \centering
    \includegraphics[width=0.7\textwidth]{oscil.png}
    \caption{Oscilloscope}
    % \label{fig:Oscilloscope - Tektronix TBS1072B}
\end{figure}

\subsection*{Signal Generator}
\addcontentsline{toc}{subsection}{Signal Generator}
A signal generator is an electronic device that produces repeating or non-repeating waveforms. These waveforms can be of different shapes, such as sine waves, square waves, or triangular waves, and are used to test and troubleshoot electronic circuits. The properties of the waves generated by it can be tweaked to match the purpose.\\\\
\textbf{Key Features of a Signal Generator:}
\begin{itemize}
    \item \textbf{Wave Generation:} Creates various types of waveforms or signals.
    \item \textbf{Frequency Adjustment:} Frequency can be adjusted as required.
    \item \textbf{Modulation:} Modulate amplitude, frequency, or phase of signals.\\
\end{itemize}
\textbf{Types:}
\begin{itemize}
    \item \textbf{Function Generators:} Generate standard waveforms like sine, square, and triangle.
    \item \textbf{Arbitrary Waveform Generators:} Create user-defined waveforms.
    \item \textbf{RF Signal Generators:} Used for radio frequencies.
\end{itemize}

\begin{figure}[H]
    \centering
    \includegraphics[width=0.7\textwidth]{sigGen.png}
    \caption{Signal Generator}
    % \label{fig:Signal Generator - SG 1639A}
\end{figure}


\section*{Required Tools \& Apparatus}
\addcontentsline{toc}{section}{Required Apparatus}
\begin{itemize}
    \item Oscilloscope - Tektronix TBS1072B
    \item Signal Generator - SG 1639A
    \item Probes
    \item Power Cables
    \item Latex
\end{itemize}

\section*{Procedure}
\addcontentsline{toc}{section}{Procedure}
\begin{enumerate}
    \item The oscilloscope \& the signal generator was turned on.
    \item The probes were connected to the oscilloscope and signal generator.
    \item The signal generator's probes were connected to the oscilloscope's.
    \item The oscilloscope was adjusted to display the signal from the signal generator.
    \item Using the modulation feature of the signal generator different types of signals was generated and the waves were observed using the oscilloscope.
\end{enumerate}

\section*{Inputs and Outputs}
\addcontentsline{toc}{section}{Inputs and Outputs}

\subsection*{Oscilloscope}
\addcontentsline{toc}{subsection}{Oscilloscope}

\subsubsection*{Display}
\begin{itemize}
    \item[] \textbf{Trigger Position Icon:} Indicates the trigger position on the screen, adjustable with the knobs.
    \item[] \textbf{Horizontal Position:} Indicates the position of the X-axis of the waveform(s).
    \item[] \textbf{Menu:} Shows various options to adjust the display for better understanding of the waveform(s).
\end{itemize}

\begin{figure}[H]
    \centering
    \includegraphics[width=0.7\textwidth]{oscilOut.png}
    \caption{Display Interface of Oscilloscope}
    % \label{fig:Display Interface of Oscilloscope}
\end{figure}

\subsubsection*{Menu}
\begin{itemize}
    \item[] \textbf{Multipurpose Knob:} Can be turned to adjust cursor position
    \item[] \textbf{Help:} Highlights entries in the Index. Highlights links in a topic. Push to select the highlighted item.
    \item[] \textbf{Math:} Scroll to position and scale the Math waveform. Scroll and push to select the operation.
    \item[] \textbf{Measure:} Scroll to highlight and push to select the type of automatic measurement for each source and various properties of that source.
    \item[] \textbf{Save/Recall:} Scroll to highlight and push to select the action and file format. Scroll through the list of files. Pressing also saves current display screenshot to a externally connected device.
    \item[] \textbf{Utility:} Scroll to highlight and push to select miscellaneous menu items. Turn to set the backlight value.
    \item[] \textbf{Autoset:} Adjusts the current display and makes it so that its shows best.
    \item[] \textbf{Run/Stop:} Pressing freezes the current output.
\end{itemize}

\begin{figure}[H]
    \centering
    \includegraphics[width=0.4\textwidth]{oscilMenu.png}
    \caption{Menu of Oscilloscope}
    % \label{fig:Menu of Oscilloscope}
\end{figure}

\subsubsection*{Horizontal Control\cite{Theworki71:online}}
\begin{itemize}
    \item[] \textbf{Position:}  Positions a waveform horizontally.
    \item[] \textbf{Scale:} Selects horizontal scale factors like time.
    \item[] \textbf{Acquire:} Displays the acquisition mode: Sample, Peak Detect, or Average.
\end{itemize}

\begin{figure}[H]
    \centering
    \includegraphics[width=0.1\textwidth]{oscilHoriz.png}
    \caption{Horizontal Control of Oscilloscope}
    % \label{fig:Horizontal Control of Oscilloscope}
\end{figure}

\subsubsection*{Vertical Control\cite{Theworki71:online}}
\begin{itemize}
    \item[] \textbf{Position (1 \& 2):} Positions a waveform vertically.
    \item[] \textbf{Menu (1 \& 2):} Displays the Vertical menu selections and toggles the display of the channel waveform on and off.
    \item[] \textbf{Scale (1 \& 2):} Selects vertical scale factors such as amplitude.
\end{itemize}

\begin{figure}[H]
    \centering
    \includegraphics[width=0.3\textwidth]{oscilVert.png}
    \caption{Vertical Control of Oscilloscope}
    % \label{fig:Vertical Control of Oscilloscope}
\end{figure}

\subsubsection*{Trigger Control\cite{Theworki71:online}}
\begin{itemize}
    \item[] \textbf{Trigger Menu:}
          \begin{itemize}
              \item Pressing the button once will open the Trigger Menu.
              \item Holding the button for more than 1.5 seconds will switch to the trigger view, where the trigger waveform replaces the channel waveform. This view allows to observe how the trigger settings (e.g., coupling) impact the trigger signal.
              \item Releasing the button will return to the standard view and stop displaying the trigger waveform.
          \end{itemize}
    \item[] \textbf{Level:}
          \begin{itemize}
              \item When using Edge or Pulse triggers, the Level knob adjusts the amplitude level at which the signal must cross to capture a waveform.
              \item Pressing the knob sets the trigger level to the vertical midpoint between the peaks of the trigger signal (50\%).
          \end{itemize}
    \item[] \textbf{Force Trig:}
          \begin{itemize}
              \item Use the Force Trig button to manually acquire a waveform, regardless of whether the oscilloscope detects a trigger.
              \item This function is helpful for single sequence acquisitions and Normal trigger mode.
              \item In Auto trigger mode, the oscilloscope automatically forces triggers periodically if no trigger is detected.
          \end{itemize}
\end{itemize}

\begin{figure}[H]
    \centering
    \includegraphics[width=0.1\textwidth]{oscilTrig.png}
    \caption{Trigger Control of Oscilloscope}
    % \label{fig:Trigger Control of Oscilloscope}
\end{figure}

\subsubsection*{Connectors}
\begin{itemize}
    \item[] \textbf{Blue Port:} Used to input one of the waveform signals for display.
    \item[] \textbf{Yellow Port:} Used to input the second waveform signal for display.
    \item[] \textbf{Ext Trig (External Trigger):} Provides a connection for an external trigger signal to synchronize waveform acquisition.
\end{itemize}

\begin{figure}[H]
    \centering
    \includegraphics[width=0.5\textwidth]{oscilProbes.png}
    \caption{Input Connectors of Oscilloscope}
    % \label{fig:Input Connectors of Oscilloscope}
\end{figure}

\subsection*{Signal Generators}
\addcontentsline{toc}{subsection}{Signal Generators}

\begin{figure}[H]
    \centering
    \includegraphics[width=0.8\textwidth]{sigGen2.png}
    \caption{Signal Generator}
    % \label{fig:Signal Generator}
\end{figure}

\begin{enumerate}
    \item \textbf{Power:} Turns the device "ON" when the button is pressed
    \item \textbf{Frequency Adjust:} Adjusts the output frequency.
    \item \textbf{Select Switch (INT/EXT):} \cite{signal}Measures external (EXT) frequency when the button is pressed.
    \item \textbf{Attenuation (Input):} Reduces input signal by 10dB when pressed.
    \item \textbf{Input Socket:} \cite{signal}Socket for external (EXT) signal input.
    \item \textbf{SYMM. Switch:} Adjusts the waveform to be symmetrical when the switch is pulled out.
    \item \textbf{VCF Input Socket:} \cite{signal}Allows external signals to control the main output frequency.
    \item \textbf{DC Offset Switch:} \cite{signal}Adjusts the DC offset when the switch is pulled out.
    \item \textbf{TTL Output:} \cite{signal}Socket for TTL signal output.
    \item \textbf{Amplitude:}Adjusts the output signal's amplitude (inverts when the switch is pulled out).
    \item \textbf{Output Socket:} Socket for the main signal output.
    \item \textbf{Attenuation (Output):} Reduces output signal by 40dB when pressed.
    \item \textbf{Attenuation (Output):} Reduces output signal by 20dB when pressed.
    \item \textbf{Waveform Selector:} Selects the output waveform type.
    \item \textbf{Frequency Range:} Adjusts and selects the output frequency range.
    \item \textbf{Frequency Counter:} Displays the frequency of the output signal.
\end{enumerate}

\section*{Discussion}
\addcontentsline{toc}{section}{Discussion}
In this experiment, we worked with two essential tools in electronics laboratories: the oscilloscope and the signal generator. The oscilloscope allows us to visualize electronic signals, while the signal generator produces a variety of signals at adjustable frequencies. Together, these devices play a critical role in testing and analyzing electronic circuits.\\\\
We gained hands-on experience by using these tools to observe and evaluate signals. Adjustments were made to both devices to see how changes in settings affected the signals. Additionally, we connected the signal generator to the oscilloscope to monitor the generated waveforms directly on the screen.\\\\
This practical session was highly beneficial, providing us with valuable experience in using these tools effectively. It demonstrated their importance in the field of electronics for testing, troubleshooting, and analyzing circuits.\\\\
In summary, the oscilloscope and signal generator are indispensable instruments in any electronics laboratory. The oscilloscope offers a detailed representation of signal characteristics, while the signal generator provides controlled test signals. By combining their functionalities, they allow engineers to create, measure, and analyze signals, ensuring that electronic devices and circuits perform as expected.

\section*{Precaution \& Conclusion}
\addcontentsline{toc}{section}{Precaution}
To ensure safe and effective use of the equipment, the following precautions were observed during the experiment:
\begin{itemize}
    \item Both the oscilloscope and signal generator were powered off when not in use to prevent potential damage.
    \item Probes and cables were connected or disconnected only when the devices were switched off to maintain safety and avoid electrical faults.
    \item Settings on the instruments were adjusted carefully to prevent any misconfigurations or damage to the devices.
    \item Only the provided, calibrated probes and cables were used to ensure accurate measurements.
    \item After completing the experiment, all equipment was returned to its designated safe storage location.\\\\
\end{itemize}
This experiment reinforced the critical role of the oscilloscope and signal generator in electronics. It highlighted their complementary functionalities, with the oscilloscope visualizing signal behavior and the signal generator providing precise, controllable inputs for testing. Through this exercise, we developed a deeper understanding of how these tools are used in real-world applications to diagnose, validate, and optimize electronic circuits. Moreover, adhering to proper precautions ensured the longevity of the equipment and the safety of the users, emphasizing the importance of best practices in a laboratory setting.

\bibliographystyle{IEEEtran}
\renewcommand{\bibname}{References}
\addcontentsline{toc}{section}{References}
\bibliography{ref}

\end{document}
